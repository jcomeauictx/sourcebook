% adapted from https://www.overleaf.com/learn/latex/Sections_and_chapters
\documentclass{article}
% sloppy or emergencystretch needed for markdown processor
\sloppy  % helps READMEs and listings lines wrap
\usepackage[letterpaper,portrait,margin=1in]{geometry}
\usepackage{parskip}  % vertical spacing rather than indentation of paragraphs
\usepackage{xstring}  % string length (\StrLen) for ifthenelse actions
\usepackage[pipeTables,tableCaptions]{markdown}
\usepackage{verbatim} % input a file verbatim, e.g., don't render as TeX
\usepackage{listings} % source code listings
\usepackage{xcolor}  % for colored syntax highlighting
\usepackage{CJK}  % CJK characters
%\renewcommand*{\ttdefault}{lmvtt}
\title{$CAPTION}
\begin{document}
\begin{CJK}{UTF8}{}
\maketitle
\lstset{
    numbers=left,                % line numbers on the left
    numberstyle=\small,          % style of line numbers
    showstringspaces=false,      % do not show visible representation of spaces
    stepnumber=1,                % show every line number
    numbersep=5pt,               % gap between numbers and code
    backgroundcolor=\color{white}, % background color
    keywordstyle=\color{blue},   % keyword style
    basicstyle=\ttfamily,        % code font and (optionally) size
    frame=single,                % add a frame around the code
    breaklines=true,             % break long lines
    breakindent=0pt,             % don't indent broken part of lines
    linewidth=6.5in,             % max line length
    breakatwhitespace=false,     % breaks anywhere
    basicstyle=\ttfamily,        % not necessarily cmtt10
    extendedchars=true,          % use all possible characters
    % now define some special character representations
    % show control characters and Latin1 in different color; UTF8 normal
    literate=%
        {^^e3^^81^^a6}{{>}}1  % hiragana "te"
        {^^e3^^82^^8b}{{U}}1  % hiragana "ru"
        {^^e3^^80^^82}{{.}}1  % hiragana ".", period
        {^^e6^^84^^9b}{{<}}1  % kanji for Love
        {^^e3^^81^^97}{{+}}1  % hiragana "shi"
        {^^e6^^97^^a5}{{\symbol{"65e5}}}1  % kanji for Sun
        {^^e6^^9c^^ac}{{R}}1  % kanji for Root
        {^^c2^^a9}{{\copyright}}1  % (C) in circle
        {^^c2^^a1}{{!`}}1  % upside down exclamation point
        {^^c2^^bf}{{?`}}1  % upside down question mark
        {^^c3^^9f}{{\ss}}1  % eszett, 0xdf
        {^^c3^^a4}{{\"a}}1  % a-umlaut, 0xe4
        {^^c3^^ab}{{\"e}}1  % e-umlaut, 0xeb
        {^^c3^^b6}{{\"o}}1  % o-umlaut, 0xf6
        {^^c3^^b8}{{\o}}1  % norwegian o-slash, 0xf8
        {^^a9}{{\textcolor{gray}{\copyright}}}1
        {^^07}{{\textcolor{gray}{\^{}G}}}2  % BEL
        {^^1b}{{\textcolor{gray}{\^{}[}}}1  % ESC
        {^^a7}{{\textcolor{gray}{\S}}}1  % section symbol
        {^^a1}{{\textcolor{gray}{!`}}}1  % upside down "!"
        {^^bf}{{\textcolor{gray}{?`}}}1  % upside-down ? (latin-1)
        {^^d7}{{\textcolor{gray}{\texttimes}}}1  % times symbol
        {^^df}{{\textcolor{gray}{\ss}}}1  % eszett (latin-1)
}
\lstset{language=$LANGUAGE}
\lstinputlisting{$FILEPATH}
\end{CJK}
\end{document}
% vim: tabstop=8 expandtab shiftwidth=4 softtabstop=4
