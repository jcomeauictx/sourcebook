% adapted from https://www.overleaf.com/learn/latex/Sections_and_chapters
\documentclass{article}
% sloppy or emergencystretch needed for markdown processor
\sloppy  % helps READMEs and listings lines wrap
\usepackage[letterpaper,portrait,margin=1in]{geometry}
\usepackage{parskip}  % vertical spacing rather than indentation of paragraphs
\usepackage{xstring}  % string length (\StrLen) for ifthenelse actions
\usepackage[pipeTables,tableCaptions]{markdown}
\usepackage{verbatim} % input a file verbatim, e.g., don't render as TeX
\usepackage{listings} % source code listings
\usepackage{xcolor}  % for colored syntax highlighting
\usepackage[overlap, CJK]{ruby}  % enable Chinese/Japanese/Korean with ruby
\usepackage{CJKulem}  % "ulem" package CJK characters
\newenvironment{Japanese}{%
\CJKfamily{min}%
\CJKtilde  % not sure we want this
\CJKnospace}{}
\title{$CAPTION}
\begin{document}
\begin{CJK}{UTF8}{}
\begin{Japanese}
\maketitle
\lstset{
    numbers=left,                % line numbers on the left
    numberstyle=\small,          % style of line numbers
    showstringspaces=false,      % do not show visible representation of spaces
    stepnumber=1,                % show every line number
    numbersep=5pt,               % gap between numbers and code
    backgroundcolor=\color{white}, % background color
    keywordstyle=\color{blue},   % keyword style
    basicstyle=\ttfamily,        % code font and (optionally) size
    frame=single,                % add a frame around the code
    breaklines=true,             % break long lines
    breakindent=0pt,             % don't indent broken part of lines
    linewidth=6.5in,             % max line length
    breakatwhitespace=false,     % breaks anywhere
    % now define some special character representations
    % show control characters and Latin1 in different color; UTF8 normal
    literate={^^1b}{{\textcolor{green}{\^{}[}}}1
     {^^c2^^a9}{{\copyright}}2
     {^^a9}{{\textcolor{green}{\copyright}}}1
}
\lstset{language=$LANGUAGE}
\lstinputlisting{$FILEPATH}
\end{Japanese}
\end{CJK}
\end{document}
